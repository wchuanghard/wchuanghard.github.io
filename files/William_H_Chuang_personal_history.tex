\documentclass[10pt]{amsart}
\usepackage{amssymb, amscd, amsmath, amsthm}
\usepackage{hyperref}
\usepackage{graphicx}
%%%%%%%%%%%%% box
\usepackage{tcolorbox}

\newtheorem*{utheorem}{Theorem}
\newtheorem*{ucorollary}{Corollary}
\newtheorem{theorem}{Theorem}
\newtheorem{lemma}{Lemma}
\newtheorem*{ulemma}{Lemma}
\newtheorem{definition}[theorem]{Definition}
\thispagestyle{empty}

\usepackage{hyperref}
\usepackage[margin=1in]{geometry}

\begin{document}

\title{Personal History}
\author{William H. Chuang}
\date{\today}
\maketitle


%I have always wanted to pursue a PhD and be a Doctor just like my dad. When I was little, I was the first witness to see him jump out of the chair from piles of papers and books and thrilled to tell my mum of his discoveries. Most of the time, my father is a calm and quiet person. Only for those moments, he seemed excited and energetic. Later I learned those were the aha moments and they are rare and precious. PhD and Doctor is a long journey that requires focus, accumulation, and responsibility. It is not just personal pursuit; it also comes with educational responsibilities.

I have always aspired to pursue a PhD and become a professor just like my father. When I was a kid, I saw him jump out of his chair from mounds of papers and books, thrilled to tell my mother about his discovery. My father is, for most of the time, a calm and quiet individual. He appeared excited and energetic only for those few moments. I guess that is the nascent stage of my aspiration to pursue a PhD, to experience the same joy and excitement just like my dad did. Later on, I realized that those were the aha moments, which are extremely rare and valuable. A doctorate degree is the culmination of a long path that necessitates concentration, accumulation, and accountability. %With the path I have taken in Mathematics and who I am today, I am confident that I am prepared for all of them to be a good candidate for the PhD program at the University of Arizona.



It took some time and a few turns for me to realize Mathematics is something I genuinely enjoy and want to pursue. 

It was the fall of 2007 when I first started college at National Dong Hwa University (NDHU) in Taiwan, and my major was Physics. Within two and a half years, I had completed major required courses and was hired by the Physics Department to teach and assist my colleagues with difficult problems. I took a medical leave of absence from NDHU in the spring of 2010 due to pneumothorax and returned to Taipei to live with my parents for better recovery. After passing the admission exam, which was deemed the most difficult exam at the time with a less than 4$\%$ acceptance rate, I transferred to National Taiwan University (NTU), Taiwan's top-ranked university with the first class hospital, to live with my family and receive better medical care. I had two relapses during 2011 and 2013, and had to take a medical leave of absence from NTU until I was fully recovered. Throughout these medical leaves, I did thousands of problems in calculus and physics. It was also during that time I understood the link between physics and mathematics, and how mathematics could elevate physics study.



My interest in proving theorems stemmed from Prof. Paul Zeitz's problem solving course based on his book ``The Art and Craft of Problem Solving''. It was to uncover a pattern in a number of areas in two-dimensional Euclidean space partitioned by $n$ lines. Although I studied math in Taiwan, my main concentration was on how to apply mathematics to real-world problems. With the help of Prof. Zeitz's problem solving strategies, for the first time, I came up with a proof on my own that is not intended to serve a real-world problem. At that very moment, I felt delighted and thrilled! Since then, I have been taking more and more pure math classes.


In the fall of 2017, I was accepted into the MASS program at Pennsylvania State University. I participated in three expository research projects. Meanwhile, I enrolled in MASS honor classes that were beyond my level, including algebra (Elliptic Curves and Elliptic Functions), analysis (Functional Analysis), geometry (Knot Theory), seminars, and colloquia. I learned a lot from the program's faculty, particularly Prof. Sergei Tabachnikov. His classes exposed me to knot theory, real and complex analysis, Galois theory, differential geometry, symplectic geometry, and special linear groups. The most thrilling moment came at the conclusion of the semester, when he demonstrated the smallest working example of Kontsevich’s integral to construct the universal Vassiliev invariant of a knot. Every week, we were given somewhere between five to ten new problems that were completely original and had never been published before. Looking back, I realize how much I value this curriculum. For the first time, I experienced what it was like to be a mathematician, and I became attracted to the contemporary growth of pure math after starting a research project with hands-on and innovative challenges.


When I returned to San Francisco, I had my first mathematical research. In the spring of 2018, I had the opportunity to conduct an independent study on Analytic Number Theory with Prof. Zeitz, utilizing Jameson's The Prime Number Theorem and two other references: Ahlfors' Complex Analysis and The Prime Numbers and Their Distribution by G. Tenenbaum and M. M. France. Every week, I met with Prof. Zeitz once, sometimes twice, to discuss my progress in reconstructing the proof of the Prime Number Theorem. Prof. Zeitz taught me how to examine and grasp diverse points of view, and during this independent study, I also developed a technique for reading a math book. That is, swiftly read it several times until I know the book backwards and forwards, including its definitions, theorems, and proofs. Then, read it a couple more times with the objective of seeing a picture in which the complete theory given in the book can emerge. In other words, with that picture in mind, I can reconstruct the entire book. After the semester, I worked on a self-reading side project based on M. Stein's Complex Analysis with the help of Serge Lang's Introduction to Modular Forms and Elliptic Curves.



To further my understanding of mathematics, I enrolled in an MA degree program in mathematics at San Francisco State University in the spring of 2019. I want to gain more mathematics skills and learn how to communicate and teach math. The next semester, I was given the opportunity to conduct independent research with Dr. Joseph Gubeladze. While Dr. Gubeladze illustrated how he solved problems and established theorems, at the end of that semester I realized that I was finally capable of accomplishing it on my own, even when he presented me with his original challenges. At the same time, I started my first college teaching job in the United States and took Dr. Kim Seachore's teaching workshop. Dr. Seashore demonstrated what it takes to be an effective college educator. Because I was born and grew up in Asia and did not grasp American culture from the roots up, having an entire class of domestic graduate teaching assistants as classmates to share teaching experience throughout the first semester I began teaching and having a professor like Dr. Seashore to help me adapt to the cultural difference was extremely beneficial. I also learnt that it is vital to enable students to freely explore mathematics in group activities and to let the group dynamics flow in order to encourage their interest and turn them into spontaneous learners.


I took real analysis II with Dr. Alex Schuster in the spring of 2020. It was a highly useful guide, since he offered a systematic method for dealing with infinite series, based on William Wade's book. Later, while I worked on my thesis, I discovered how handy this tool box was. The most significant lesson I learnt from Dr. Schuster was not just how to utilize a tool but also why it works.

Dr. Schuster recommends Dr. Emily Clader's topology course as the next real analysis II course. It was amazing how everything in topology could be taught in such a brilliant way that even students who have not taken real analysis can comprehend it. Under her guidance, I learnt how to mathematically communicate with others by sharing our proofs and having group discussions.

The following course of real analysis II recommended by Dr. Schuster is Dr. Emily Clader’s topology course. It was astonishing that everything in topology can be introduced in such a nice way that even for some students who have taken real analysis, they can still understand it. Through her teaching, I learned to write a proof to communicate with others, and it should be a pleasant and enjoyable experience.

I had the opportunity to attend Dr. Sheldon Axler's measure theory course in the fall of 2020,
followed by his functional analysis course in the spring of 2020. His new book ``Measure, Integration, $\&$ Analysis'' was the basis for those classes. This experience allowed me to directly examine how an established mathematician builds everything described in his book from the ground up. Within these two classes, I discovered that I got increasingly interested in constructing conjectures and generating new conceptions or techniques for proving theorems.

Math is about raising questions and forming conjectures in order to see a single picture from which the entire theory emerges. Throughout this semester, I had an opportunity to work with Dr. Chun-Kit Lai on an independent project on the estimation of the Hausdorff dimension of the limit set of Kleinian groups. It was then expanded into my thesis. I learned how to formulate questions and hypotheses that could lead to new mathematical conclusions. The tactics repeatedly failed to enter the heart of the challenge, which was to directly compute Hausdorff dimension even for the simplest case, in the first nine months. However, each failure pushed me closer to an epiphany that eventually led to a solution to the fundamental problem of why my prior strategies failed. By removing each root cause, a visual approach for directly computing the Hausdorff dimension of some specified Schottky groups was devised. Furthermore, I learnt how to completely create a publishable proof by organizing numerous components and creating multiple clear examples down to the specifics during the thesis writing process.


All of them gradually shaped me into who I am today and my interest in Mathematics has grown day by day, inspiring me to pursue a PhD in mathematics and to become qualified to make socially meaningful contributions as a professor at college or university.



\end{document}

